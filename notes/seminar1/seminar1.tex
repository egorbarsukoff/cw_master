\documentclass[unicode]{beamer}
% If you have more than three sections or more than three subsections in at least one section,
% you might want to use the [compress] switch. In this case, only the current (sub-) section
% is displayed in the header and not the full overview.
%\mode<presentation>
%{
%  
%
%  \setbeamercovered{transparent}
%  \beamertemplatenavigationsymbolsempty
%  % or whatever (possibly just delete it)
%}

\mode<handout> {
    \usepackage{pgfpages}
    \setbeameroption{show notes}
    \pgfpagesuselayout{2 on 1}[a4paper, border shrink=5mm]
}

%\usepackage{pscyr}
\usepackage[T2A]{fontenc}
\usepackage[utf8]{inputenc}
\usepackage[russian]{babel}
\usepackage{amsthm}
\usepackage{pdfpages}
\usepackage{bibentry}

\usepackage{natbib}   % omit 'round' option if you prefer square brackets
\bibliographystyle{plainnat}

\graphicspath{{fig/}}

%\usepackage{tikz}
% you only need this when using TikZ graphics

%\newtheorem{theorem}{Теорема}
%\newtheorem{example}{Пример}
%\newtheorem{definition}{Определение}


\DeclareMathOperator*{\E}{\mathrm{E}}
\DeclareMathOperator*{\D}{\mathrm{D}}
\DeclareMathOperator*{\sign}{sign}
\DeclareMathOperator*{\argmax}{arg\,max}
\DeclareMathOperator*{\argmin}{arg\,min}
\DeclareMathOperator*{\Int}{Int}
\DeclareMathOperator*{\err}{err}
\newcommand\norm[1]{\left\lVert#1\right\rVert}
\newcommand\abs[1]{\left\lvert#1\right\rvert}


\title[Оптимальные планы для оценивания производных]{How ambition affects role model choices: heterogeneous peer-effects in networks}

%\author{Барсуков Егор Вячеславович, гр. 16-Б.04мм}
%\institute[СПбГУ]{
%    Санкт-Петербургский государственный университет \\
%    Математико-Механический факультет \\
%    Кафедра статистического моделирования \\
%   \vspace{1.25cm}
%    Преддипломная практика
%}
%
%\date{
%	Санкт-Петербург \\
%    2020 г.
%}

%\subject{Beamer}
% This is only inserted into the PDF information catalog. Can be left
% out.

% Delete this, if you do not want the table of contents to pop up at
% the beginning of each subsection:
% \AtBeginSubsection[]
% {
%   \begin{frame}<beamer>
%     \frametitle{Outline}
%     \tableofcontents[currentsection,currentsubsection]
%   \end{frame}
% }
%
%\institute[Санкт-Петербургский Государственный Университет]{
%    \small
%    Санкт-Петербургский государственный университет\\
%    Прикладная математика и информатика\\
%    Вычислительная стохастика и статистические модели\\
%    \vspace{1.25cm}
%    Преддипломная практика}

\date{Санкт-Петербург, 2021}

\begin{document}
\begin{frame}
    \titlepage
\end{frame}

\begin{frame}
\frametitle{Предпосылки}
	$$
	x_{ig} = z_{ig} + \gamma_g \gamma + \frac{\theta}{N_g - 1} \sum\limits_{j \neq i} x_{jg} + \varepsilon_{ig}
	$$
\end{frame}

\begin{frame}
\frametitle{В предыдущих сериях}
	$$
	u_i = -\frac{1}{2}(x_i-\alpha_i)^2-\frac{1}{2}\frac{\lambda}{1-\lambda}\left(x_i - \overline{x}_i\right)^2
	$$
	F.O.C.:
	$$
	x_i = (1- \lambda)\alpha_i + \lambda \bar{x}_i
	$$
	Предполагалось,  что $\bar{x}_i$ --- среднее по всем соседям в заданном графе.
\end{frame}

\begin{frame}
\frametitle{Вопрос}
	Что, если $\bar{x}_i$ --- какое-то другое среднее? Если структура графа определяется внутренними свойствами выборки?
	
	Пусть $\bar{x}_i$ зависит от какого-то параметра $\beta_i$,  причем:
	\begin{itemize}
	\item $\bar{x}_i(0) = \mathbb{E}x$
	\item $\lim\limits_{\beta \to \infty} \bar{x}_i(\beta) = \argmax x$
	\item $\lim\limits_{\beta \to -\infty} \bar{x}_i(\beta) = \argmin x$
	\item монотонно возрастает по $\beta$
	\end{itemize}
\end{frame}

\begin{frame}
\frametitle{Социальная норма}
$$
\bar{x}_i = \frac{\sum x_j^{\beta_i + 1}}{\sum x_j^{\beta_i}}
$$
\begin{itemize}
	\item зависит от $\beta_i$ и $x$
	\item можно записать как взвешенное среднее (поэтому задает структуру взвешенного графа)
	
\end{itemize}
\end{frame}

\begin{frame}
\frametitle{Fixed - point condition}
	$$
	x_i = (1 - \lambda) a_i + \lambda \frac{\sum x_j^{\beta_i + 1}}{\sum x_j^{\beta_i}}
	$$
\end{frame}

\begin{frame}
	Что было сделано:
	\begin{itemize}
	\item Показано существование и единственность решений для F.O.C.  с (не только) предложенным $\bar{x}$
	\item Показаны свойства $\bar{x}$ в непрерывном случае
	\item Разработан численный алгоритм
	\end{itemize}
\end{frame}

\begin{frame}
	 Что нужно сделать
	 \begin{itemize}
	 \item Разбор литературы
	 \item Численные результаты
	 \item ???
	 \end{itemize}
\end{frame}

\end{document}
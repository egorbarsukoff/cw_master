 

\documentclass[12pt]{article}
\usepackage[russian]{babel}
%%%%%%%%%%%%%%%%%%%%%%%%%%%%%%%%%%%%%%%%%%%%%%%%%%%%%%%%%%%%%%%%%%%%%%%%%%%%%%%%%%%%%%%%%%%%%%%%%%%%%%%%%%%%%%%%%%%%%%%%%%%%%%%%%%%%%%%%%%%%%%%%%%%%%%%%%%%%%%%%%%%%%%%%%%%%%%%%%%%%%%%%%%%%%%%%%%%%%%%%%%%%%%%%%%%%%%%%%%%%%%%%%%%%%%%%%%%%%%%%%%%%%%%%%%%%
%\usepackage{setspace}
\usepackage{amssymb}
\usepackage{amsfonts}
\usepackage{amsmath}
\usepackage[nohead]{geometry}
\usepackage{indentfirst}
\usepackage{graphicx}
\usepackage{hyperref}
\usepackage{float}
\usepackage{geometry}
\usepackage{xcolor}






\setcounter{MaxMatrixCols}{10}


\newtheorem{theorem}{Теорема}
\newtheorem{acknowledgement}{Acknowledgement}
\newtheorem{algorithm}[theorem]{Algorithm}
\newtheorem{axiom}[theorem]{Axiom}
\newtheorem{case}[theorem]{Case}
\newtheorem{claim}[theorem]{Claim}
\newtheorem{conclusion}[theorem]{Conclusion}
\newtheorem{condition}[theorem]{Condition}
\newtheorem{conjecture}[theorem]{Conjecture}
\newtheorem{corollary}[theorem]{Corollary}
\newtheorem{criterion}[theorem]{Criterion}
\newtheorem{definition}[theorem]{Definition}
\newtheorem{example}[theorem]{Example}
\newtheorem{exercise}[theorem]{Exercise}
\newtheorem{lemma}{Lemma}
\newtheorem{notation}[theorem]{Notation}
\newtheorem{problem}[theorem]{Problem}
\newtheorem{proposition}{Предложение}
\newtheorem{remark}[theorem]{Remark}
\newtheorem{solution}[theorem]{Solution}
\newtheorem{summary}[theorem]{Summary}
\newtheorem{assumption}{Assumption}
%\newtheorem{corollary}{Corollary}[section]
\newenvironment{proof}[1][Доказательство]{\noindent\textbf{#1.} }{\ \rule{0.5em}{0.5em}}
\geometry{left=0.75in,right=0.75in,top=1in,bottom=1in}

\newcommand\norm[1]{\left\lVert#1\right\rVert}
\newcommand\abs[1]{\left\lvert#1\right\rvert}

\DeclareMathOperator*{\esssup}{ess\,sup}
\DeclareMathOperator*{\essinf}{ess\,inf}


\begin{document}
Пусть каждый индивид определяется каким-то вектором параметров $\theta \in \mathcal{A}$.  Пусть эти параметры имеют распределение $G$,  то есть вероятность того,  что $\theta$ принадлежит какому-то множеству $A$ определяется следующим образом:
$$
P_G(A) = \int\limits_A G(d \theta),
$$
при этом $P_G(\mathcal{A}) =1$.
 
Пусть $\tilde{\omega}(\theta', \theta'') \geqslant 0$ --- вес,  показывающий влияние индивида $\theta''$ на $\theta'$ при составлении социальной нормы.  Чтобы оценивать среднее, а не агрегированное влияние на индивида,  введем нормированные веса $\omega$:
$$
\omega(\theta', \theta'') = \frac{\tilde{\omega}(\theta', \theta'')}{\int\limits_{\theta \in \mathcal{A}} \tilde{\omega}(\theta, \theta'')G(d \theta)} .
$$
Тогда социальная норма $\overline{x}(\theta)$ описывается как
$$
\overline{x}(\theta) = \int\limits_{\theta' \in \mathcal{A}} x(\theta') \omega(\theta, \theta') G(d \theta').
$$
Так как по определению $\omega$
$$
\int\limits_{\theta' \in \mathcal{A}} \omega(\theta, \theta') G(d \theta') = 1 \quad \forall \theta \in \mathcal{A},
$$
то
$$
G_\theta (A) = \int\limits_{\theta' \in A} \omega(\theta, \theta') G(d \theta')
$$
определяет распределение для каждого индивида. Его можно интерпретировать как распределение внимания индивида $\theta$ при составлении социальной нормы.  Тогда

$$
\overline{x}(\theta) =  \int\limits_{\theta' \in \mathcal{A}} x(\theta') G_\theta(d \theta').
$$

Пусть полезность каждого индивида определяется следующим образом:

\begin{equation*}
u(\theta) = -\frac{1}{2}(x(\theta)-\alpha(\theta))^2-\frac{1}{2}\frac{\lambda}{1-\lambda}\left(x(\theta) - \overline{x}(\theta)\right)^2,
\end{equation*}
где $\lambda \in (0, 1)$.  Условие первого порядка (для $x$)записывается как
\begin{equation}
\label{eq: x}
x(\theta) = (1-\lambda)\alpha(\theta) + \lambda \overline{x}(\theta).
\end{equation}


\begin{proposition}
Существует единственная функция $x(\theta)$ удовлетворяющая условию \eqref{eq: x}.
\end{proposition}
\begin{proof}
Пусть $\rho(x', x'') = \norm{x' - x''}_\infty = \sup\limits_\theta \abs{x'(\theta) - x''(\theta)}$ --- метрика,  порожденная sup-нормой. Также пусть $\mathbb{A} = (1-\lambda)\alpha(\theta) + \lambda \overline{x}(\theta)$ --- оператор,  воздействующий, как правая часть \eqref{eq: x}.  Для того,  чтобы у \eqref{eq: x} существовала единственная неподвижная точка,  достаточно показать, что $\mathbb{A}$ --- сжимающее отображение,  т.е.  существует такое $\alpha \in (0, 1)$, что $\rho(\mathbb{A}x',\mathbb{A}x'') \leqslant \alpha \rho(x', x'')$ для любых $x', x''$.

\begin{align*}
\rho(\mathbb{A}x',\mathbb{A}x'') &= \norm{(1-\lambda)\alpha + \lambda \overline{x'} - ((1-\lambda)\alpha + \lambda \overline{x''})}_\infty \\
&= \lambda \norm{\overline{x'} - \overline{x''}}_\infty 
= \lambda \sup\limits_\theta \abs{\overline{x'}(\theta) - \overline{x''}(\theta)} \\
&= \lambda \sup\limits_\theta \abs{\int\limits_{\theta' \in \mathcal{A}} x'(\theta') G_\theta(d \theta') - \int\limits_{\theta' \in \mathcal{A}} x''(\theta') G_\theta(d \theta')} \\
&= \lambda \sup\limits_\theta \abs{\int\limits_{\theta' \in \mathcal{A}} (x'(\theta') - x''(\theta')) G_\theta(d \theta')} \\
&\leq \lambda \sup\limits_\theta \abs{\int\limits_{\theta' \in \mathcal{A}} \abs{x'(\theta') - x''(\theta')} G_\theta(d \theta')} \\
&\leq \lambda \sup\limits_\theta \abs{\int\limits_{\theta' \in \mathcal{A}} \norm{x' - x''}_\infty G_\theta(d \theta')} \\
&= \lambda \sup\limits_\theta \abs{\int\limits_{\theta' \in \mathcal{A}} \rho(x', x'') G_\theta(d \theta')} \\
&= \lambda \sup\limits_\theta \abs{\rho(x', x'')} = \lambda \rho(x', x'')
\end{align*}

Но $\lambda < 1$,  поэтому $\mathbb{A}$ --- сжимающее отображение.  Следовательно существует единственная функция $x$ удовлетворяющая \eqref{eq: x}.
\end{proof}


\begin{proposition}
\label{th:ex}
Если $\omega(\theta', \theta'') = \omega(\theta'', \theta')$,  то $\mathbb{E} x = \mathbb{E} \alpha$.
\end{proposition}
\begin{proof}
Найдем математическое ожидание от обоих частей равенства \eqref{eq: x}:
\begin{align*}
\mathbb{E}x &= \mathbb{E} \left[(1-\lambda)\alpha(\theta) + \lambda \overline{x}(\theta) \right] \\
&= (1-\lambda) \mathbb{E} \alpha + \lambda \mathbb{E}  \overline{x}.
\end{align*}
Вычислим $\mathbb{E} \overline{x}$:
\begin{align*}
\mathbb{E} \overline{x} &= \mathbb{E} \int\limits_{\theta' \in \mathcal{A}} x(\theta') \omega(\theta, \theta') G(d \theta') \\
&=  \int\limits_{\theta' \in \mathcal{A}} x(\theta') \mathbb{E} \left[ \omega(\theta, \theta') \right] G(d \theta') \\ 
&=  \int\limits_{\theta' \in \mathcal{A}} x(\theta') \mathbb{E} \left[ \omega(\theta', \theta) \right] G(d \theta') \\
&=  \int\limits_{\theta' \in \mathcal{A}} x(\theta') G(d \theta') = \mathbb{E} x.
\end{align*}
Тогда 
\begin{align*}
\mathbb{E}x &= (1-\lambda) \mathbb{E} \alpha + \lambda \mathbb{E} x,\\
(1-\lambda)\mathbb{E}x &=(1-\lambda) \mathbb{E}\alpha, \\
\mathbb{E}x &=  \mathbb{E}\alpha.
\end{align*}
\end{proof}

Пусть 
$$
\tilde{\omega}(\theta', \theta'') = x(\theta'')^{\beta(\theta')},
$$
тогда нормированные веса равны
\begin{equation}
\label{eq:weight}
\omega(\theta', \theta'') = \frac{x(\theta'')^{\beta(\theta')}}{\int\limits_{\theta \in \mathcal{A}} x(\theta)^{\beta(\theta')}G(d \theta)}.
\end{equation}

\begin{proposition}
Пусть веса заданы как \eqref{eq:weight} и $x(\theta) \geq 0$ для любого $\theta$.  Тогда 
\begin{enumerate}
\item $\lim\limits_{\beta \to\infty} \bar{x}(\beta) = \esssup\limits_{\theta} x(\theta)$
\item $\lim\limits_{\beta \to-\infty} \bar{x}(\beta) = \essinf\limits_{\theta} x(\theta)$.
\end{enumerate}.
\end{proposition}
\begin{proof}
Здесь будет доказано первое утверждение.  Второе доказывается абсолютно аналогичным способом.

Обозначим $\esssup x$ как $M$.   Для какого-то $\varepsilon > 0$ разобьем множество $\mathcal{A}$ на два дизъюнктных подмножества: 
$\mathcal{D}_\varepsilon = \left\{ \theta \in \mathcal{A} \mid x(\theta) > M - \varepsilon \right\}$
и 
$ \widetilde{\mathcal{D}}_\varepsilon = \mathcal{A} \setminus \mathcal{D}_\varepsilon.$
По определению $M$ множество $\mathcal{D}_\varepsilon$ имеет ненулевую меру.
Разделим интегралы в числителе и знаменателе $\bar{x}$ на два:
\begin{equation*}
\bar{x}(\theta') = 
\frac{\int\limits_{\theta \in \mathcal{A}}x(\theta)^{\beta(\theta')+1}G(d \theta)}{\int\limits_{\theta \in \widetilde{\mathcal{D}}_\varepsilon} x(\theta)^{\beta(\theta')}G(d \theta)} = 
\frac{\int\limits_{\theta \in \mathcal{D}_\varepsilon}x(\theta)^{\beta(\theta')+1}G(d \theta) + \int\limits_{\theta \in \widetilde{\mathcal{D}}_\varepsilon}x(\theta)^{\beta(\theta')+1}G(d \theta)}{\int\limits_{\theta \in \mathcal{D}_\varepsilon} x(\theta)^{\beta(\theta')}G(d \theta) + \int\limits_{\theta \in \widetilde{\mathcal{D}}_\varepsilon} x(\theta)^{\beta(\theta')}G(d \theta)}.
\end{equation*}
По теореме о среднем,  непрерывности и монотонности возведения в степень положительного числа существуют такие $x_{M_1},  x_{M_2} \in x(\mathcal{D}_\varepsilon)$ и $x_{m_1}, x_{m_2} \in x(\widetilde{\mathcal{D}}_\varepsilon)$,  что
\begin{align*}
\int\limits_{\theta \in \mathcal{D}_\varepsilon}x(\theta)^{\beta(\theta')+1}G(d \theta) &= G(\mathcal{D}_\varepsilon) x_{M_1}^{\beta + 1}  \\
\int\limits_{\theta \in \widetilde{\mathcal{D}}_\varepsilon}x(\theta)^{\beta(\theta')+1}G(d \theta) &= G(\widetilde{\mathcal{D}}_\varepsilon) x_{m_1}^{\beta+1}\\
\int\limits_{\theta \in \mathcal{D}_\varepsilon}x(\theta)^{\beta(\theta')}G(d \theta) &= G(\mathcal{D}_\varepsilon) x_{M_2}^{\beta}  \\
\int\limits_{\theta \in \widetilde{\mathcal{D}}_\varepsilon}x(\theta)^{\beta(\theta')}G(d \theta) &= G(\widetilde{\mathcal{D}}_\varepsilon) x_{m_2}^{\beta}.
\end{align*}
Поэтому 
\begin{equation*}
\bar{x}(\theta') = \frac{G(\mathcal{D}_\varepsilon) x_{M_1}^{\beta + 1} + G(\widetilde{\mathcal{D}}_\varepsilon) x_{m_1}^{\beta+1}}{G(\mathcal{D}_\varepsilon) x_{M_2}^{\beta} + G(\widetilde{\mathcal{D}}_\varepsilon) x_{m_2}^{\beta}}.
\end{equation*}
Так как $x_{M_1} > x_{m_1}$ по определению то,  при $\beta \to \infty$,  $x_{m_1} = o(x_{M_1}). $ 
Аналогично $x_{m_2} = o(x_{M_2}). $ 
Из определения чисел $x_{M_1}$ и $x_{M_2}$,  следует,  что $\abs{x_{M_i} - M} < \varepsilon$.  Поэтому при $\varepsilon \to 0$ оба $x_{M_i}$ стремятся к $M$. 
А так как $G(\mathcal{D}_\varepsilon) \neq 0$ ,  то
\begin{equation*}
\bar{x}(\theta') = \frac{G(\mathcal{D}_\varepsilon) x_{M_1}^{\beta + 1} + o(x_{M_1}^{\beta+1})}{G(\mathcal{D}_\varepsilon) x_{M_2}^{\beta} + o(x_{M_2}^{\beta})} \xrightarrow[\beta \to \infty]{ \varepsilon \to 0} \frac{G(\mathcal{D}_\varepsilon)M^{\beta+1}}{G(\mathcal{D}_\varepsilon)M^{\beta}} = M = \esssup x.
\end{equation*}
\end{proof}

Такая функция весов имеет смысл потому что при $\beta(\theta) \rightarrow \infty$ распределение $G_\theta$ слабо сходится к $\delta_{\max x}$,  а при $\beta(\theta) \rightarrow -\infty$ оно слабо сходится к $\delta_{\min x}$.  
То есть при больших параметрах $\beta$ индивид строит свою социальную норму из наиболее успешных соседей,  а при маленьких значениях $\beta$ социальной нормой для индивида будут являться наименее успешные соседи что и выражает <<амбиции>>.
Тогда \eqref{eq: x} принимает следующий вид
$$
x(\theta) = (1-\lambda)\alpha(\theta) + \lambda \frac{\int\limits_{\theta' \in \mathcal{A}} x(\theta')^{\beta(\theta) + 1} G(d \theta')}{\int\limits_{\theta' \in \mathcal{A}} x(\theta')^{\beta(\theta)}G(d \theta')}.
$$
Или,  в случае $m$ атомарных индивидов,
$$
x_i = (1-\lambda) \alpha_i + \lambda \frac{\sum\limits_{j=1}^m x_j^{\beta_i+1}}{\sum\limits_{j=1}^m x_j^{\beta_i}}, \quad i = 1,\dots , m.
$$

\end{document}